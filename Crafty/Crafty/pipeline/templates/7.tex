\documentclass{beamer}
\usepackage[utf8]{inputenc}
\usepackage{graphicx}

\usetheme{Warsaw} % A dynamic theme with a strong navigational bar on top
\usecolortheme{albatross} % A bold blue-on-dark background for high contrast
\usefonttheme{default} % Maintains Beamer's default font setting for legibility
\useinnertheme{default} % Standard inner theme to complement Warsaw's boldness
\useoutertheme{shadow} % Adds shadows to the outer elements for a 3D effect

\title{Chapter 1: Cell Injury, Cell Death, and Adaptations}
\author{Professor's Name}
\date{\today}

\begin{document}
	
	\begin{frame}
		\titlepage
	\end{frame}
	
	\begin{frame}{Outline}
		\tableofcontents
	\end{frame}
	
	\section{Introduction}
	\begin{frame}{Introduction}
	\end{frame}
	\begin{frame}{Introduction}
		\begin{itemize}
			\item Understanding cell injury, death, and adaptations is crucial for diagnosing and treating diseases.
			\item This chapter explores the mechanisms and implications of these cellular processes.
		\end{itemize}
	\end{frame}
	
	\section{Necrosis}
	\begin{frame}{Necrosis}
	\end{frame}
	\begin{frame}{Necrosis}
		\begin{itemize}
			\item Necrosis is a form of cell death characterized by cell membrane breakdown, organelle swelling, and rupture.
			\item It leads to inflammation in surrounding tissue.
		\end{itemize}
	\end{frame}
	\begin{frame}{Causes of Necrosis}
		\begin{itemize}
			\item Caused by external factors like toxins, infections, or trauma.
		\end{itemize}
	\end{frame}
	\begin{frame}{Types of Necrosis}
		\begin{itemize}
			\item Types include coagulative, liquefactive, caseous, and fat necrosis.
		\end{itemize}
	\end{frame}
	\begin{frame}{Example of Necrosis}
		\begin{itemize}
			\item \textbf{Example:} Coagulative necrosis often occurs in the heart after a myocardial infarction, where lack of oxygen leads to cell death.
		\end{itemize}
	\end{frame}
	
	\section{Apoptosis}
	\begin{frame}{Apoptosis}
	\end{frame}
	\begin{frame}{Apoptosis}
		\begin{itemize}
			\item Apoptosis is programmed cell death, crucial for removing damaged or unnecessary cells.
			\item Characterized by cell shrinkage, chromatin condensation, and apoptotic bodies formation.
		\end{itemize}
	\end{frame}
	\begin{frame}{Characteristics of Apoptosis}
		\begin{itemize}
			\item Does not initiate inflammation.
		\end{itemize}
	\end{frame}
	\begin{frame}{Example of Apoptosis}
		\begin{itemize}
			\item \textbf{Example:} The elimination of webbing between fetal fingers and toes is a natural occurrence of apoptosis.
		\end{itemize}
	\end{frame}
	
	\section{Cellular Adaptations}
	\begin{frame}{Cellular Adaptations}
	\end{frame}
	\begin{frame}{Cellular Adaptations}
		\begin{itemize}
			\item Adaptations include changes in size (atrophy, hypertrophy), number (hyperplasia), form (metaplasia), and function.
		\end{itemize}
	\end{frame}
	\begin{frame}{Types of Adaptations}
		\begin{itemize}
			\item Atrophy: Decrease in cell size or number, e.g., in unused muscles.
			\item Hypertrophy: Increase in cell size, e.g., in heart muscle due to hypertension.
		\end{itemize}
	\end{frame}
	\begin{frame}{More on Adaptations}
		\begin{itemize}
			\item Metaplasia: Change of one cell type to another, e.g., in the respiratory tract of smokers.
		\end{itemize}
	\end{frame}
	\begin{frame}{Example of Adaptation}
		\begin{itemize}
			\item \textbf{Example:} Hyperplasia occurs in the endometrium during the menstrual cycle, preparing for potential pregnancy.
		\end{itemize}
	\end{frame}
	
	\section{Intracellular Accumulations}
	\begin{frame}{Intracellular Accumulations}
	\end{frame}
	\begin{frame}{Intracellular Accumulations}
		\begin{itemize}
			\item Buildup of substances cells can't use or dispose of.
			\item Examples include lipids in liver cells, proteins in kidney tubule cells, and pigments like lipofuscin.
		\end{itemize}
	\end{frame}
	\begin{frame}{Example of Intracellular Accumulations}
		\begin{itemize}
			\item \textbf{Example:} Fatty liver disease results from the accumulation of lipids in liver cells, often due to alcohol abuse or obesity.
		\end{itemize}
	\end{frame}
	
	\section{Summary}
	\begin{frame}{Summary}
	\end{frame}
	\begin{frame}{Summary}
		\begin{itemize}
			\item This chapter covered the fundamental concepts of cell injury, death, and adaptations.
			\item Understanding these processes is essential for diagnosing and managing diseases.
			\item We explored necrosis, apoptosis, cellular adaptations, and intracellular accumulations.
		\end{itemize}
	\end{frame}
	
	\begin{frame}{Thank You}
		\begin{center}
			Thank you for your attention!\\
			Questions?
		\end{center}
	\end{frame}
	
\end{document}